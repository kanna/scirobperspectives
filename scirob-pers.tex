\documentclass[12pt]{article}

% The preamble here sets up a lot of new/revised commands and
% environments.  It's annoying, but please do *not* try to strip these
% out into a separate .sty file (which could lead to the loss of some
% information when we convert the file to other formats).  Instead, keep
% them in the preamble of your main LaTeX source file.
\usepackage[font=footnotesize]{caption}
\usepackage{float}
\usepackage{epsf}
\usepackage{epsfig}
\usepackage{subfigure}
% \usepackage{subfig}
\usepackage{latexsym}
%\usepackage{algorithm}
%\usepackage[noend]{algorithmic}
% \usepackage{color}
\usepackage{color, colortbl}
\usepackage{wrapfig}
\usepackage{topcapt}
\usepackage{multirow}
\usepackage{tabularx}
\usepackage{hyperref}
\usepackage{xcolor}
\usepackage{mdwlist}

\usepackage{scicite}
\usepackage{times}

% The following parameters seem to provide a reasonable page setup.

\topmargin 0.0cm
\oddsidemargin 0.2cm
\textwidth 16cm 
\textheight 21cm
\footskip 1.0cm


%\usepackage{color}
\newcommand{\lc}[1]{\textcolor{blue}{#1}}


% \usepackage{amsmath, amsfonts, amssymb}
% \usepackage{algorithmic}
\usepackage[hmargin=1in,vmargin=1.2in]{geometry}
\usepackage{url}
\usepackage{multirow}
\usepackage[ruled,noline,linesnumbered]{algorithm2e}
\usepackage[bottom]{footmisc}
\usepackage{afterpage}
%\usepackage[caption=false]{caption}
\usepackage{eurosym}
\usepackage{enumitem}
% \usepackage{soul}
\usepackage{fancyhdr}
\usepackage{dashrule}
% \usepackage[stable]{footmisc}
% \usepackage{placeins}
% \setitemize{noitemsep,topsep=0pt,parsep=0pt,partopsep=0pt}
% \usepackage[utf8]{inputenc}
\usepackage{gensymb}
\usepackage{rotating}
\usepackage{tikz}


\widowpenalty=1000
\clubpenalty=1000

\def\rx{{\texttt{T-REX\ }}}
\def\rxe{{\texttt{T-REX}}}
\def\ls{{\texttt{LSTS\ }}}
\def\lse{{\texttt{LSTS}}}
\def\eut{{\texttt{EUROPA}$_2$\ }}
\def\eu{{\texttt{EUROPA}\ }}
\def\eue{{\texttt{EUROPA}}}
\def\eus{{\texttt{EUROPA}'s\ }}
\def\nd{{\texttt{NDDL\ }}}
\def\nde{{\texttt{NDDL}}}
\def\pd{{\texttt{PDDL\ }}}
\def\pde{{\texttt{PDDL}}}
\def\eur{{\texttt{EUROPtus\ }}}
\def\eure{{\texttt{EUROPtus}}}
\def\nas{{\texttt{NASA\ }}}
\def\nase{{\texttt{NASA}}}
\def\sml{{\texttt{SmallSat\ }}}
\def\smle{{\texttt{SmallSat}}}
\def\univ{{\texttt{UPorto\ }}}
\def\unive{{\texttt{UPorto}}}
\def\aut{{\texttt{Autonaut\ }}}
\def\aute{{\texttt{Autonaut}}}
\def\pro{{\textbf{METEOR\ }}}
\def\proe{{\textbf{METEOR}}}
\def\col{{\texttt{COLREGS\ }}}
\def\cole{{\texttt{COLREGS}}}

\def\etal{{et al.\/}}
\def\eg{e.g., }
\def\ie{{i.e.,\ }}
\def\etc{{etc.\ }}
\def\situ{{in situ \/}}
\def\PN{{\emph{PN} }}


\input{epsf}

%\usepackage{mathptmx}
%\usepackage{multirow}

\newcommand{\rtime}[1]{\par\noindent\rlap{#1} \hspace*{2.15cm}}
\newcommand{\iblank}{\par \noindent \hspace*{2.4cm} \hangindent 2.6cm}
\newcommand{\m}[1]{\ensuremath{\mathbf{#1}}}
\newcommand{\mc}[1]{\ensuremath{\mathcal{#1}}}
% \newcommand{\mb}[1]{\mbox{\boldmath$#1$\unboldmath}}
%\newcommand{\norm}[1]{\left| \left| #1 \right| \right| ^2}
\newcommand{\snr}{\hbox{SNR}}
\newcommand{\mse}{\hbox{MSE}}
\newcommand{\E}{{\mathbb E}}
\newcommand{\cn}{{\mathcal{CN}}}
\newcommand{\ba}{\begin{align*}}
\newcommand{\ea}{\end{align*}}

\newcommand{\real}{{\mathbb{R}}}
\newcommand{\integer}{{\mathbb{Z}}}
\renewcommand{\natural}{{\mathbb{N}}}
\newcommand{\argmin}{\operatorname{argmin}\displaylimits}
\newcommand{\argmax}{\operatorname{argmax}\displaylimits}

\newcommand{\relthresh}{{T_{\text{rel}}}}
\newcommand{\absthresh}{{T_{\text{abs}}}}

\newcommand{\nprof}{{N_{\text{prof}}}}
\newcommand{\DM}{{DM}}
\newcommand{\UM}{{UM}}
\newcommand{\deltaMax}{{\partial_{\max}}}
\newcommand{\IFD}{{IFD}}
\newcommand{\IFU}{{IFU}}

\newcommand{\mvdiff}{\mathbf{mvd}}
\newcommand{\mvest}{\widehat{\mvdiff}}
\newcommand{\prof}{p}

\newtheorem{Prop}{Proposition}
\newtheorem{Theorem}{Theorem}
\newtheorem{Lemma}{Lemma}
\newtheorem{Corrolary}{Corollary}

\def\be{\begin{equation}}
\def\ee{\end{equation}}

\newlength{\doublespacelength}
\setlength{\doublespacelength}{\baselineskip}
\addtolength{\doublespacelength}{0.5\baselineskip}
\newcommand{\doublespace}{\setlength{\baselineskip}{\doublespacelength}}

\newlength{\singlespacelength}
\setlength{\singlespacelength}{\baselineskip}
\newcommand{\singlespace}{\setlength{\baselineskip}{\singlespacelength}}


\newlength{\savedspacing}
\newcommand{\savespacing}{\setlength{\savedspacing}{\baselineskip}}
\newcommand{\restorespacing}{\setlength{\baselineskip}{\savedspacing}}

\setlength{\parskip}{0pt}
\setlength{\parsep}{0pt}
\setlength{\headsep}{0pt}
\setlength{\topskip}{0pt}
\setlength{\topmargin}{0pt}
\setlength{\topsep}{0pt}
\setlength{\partopsep}{0pt}
% \setlength{\parindent}{0pt}

\newcommand{\unit}[1]{\ensuremath{\mathrm{#1}}}                  %%%% to units and other roman math stuff
% \linespread{0.98}
% % \linespread{2.00}

\newcounter{quotenumber}

\newenvironment{numquote}{%
    \begin{enumerate}%
     \setcounter{enumi}{\value{quotenumber}}%
     \color{darkgray}
    \item \begin{quote}%
}{%
    \end{quote}%
    \setcounter{quotenumber}{\value{enumi}}
    \end{enumerate}%
}%

\makeatletter
\def\myitem{%
   \@ifnextchar[ \@myitem{\@noitemargtrue\@myitem[\@itemlabel]}}
\def\@myitem[#1]{\item[#1]\mbox{}}
\makeatother



\newcommand\blankpage{%
    \null
    \thispagestyle{empty}%
    \addtocounter{page}{-1}%
    \newpage}

\setcounter{secnumdepth}{0} 

\let\oldthebibliography\thebibliography
\let\endoldthebibliography\endthebibliography
\renewenvironment{thebibliography}[1]{
  \begin{oldthebibliography}{#1}
    \setlength{\itemsep}{0em}
    \setlength{\parskip}{0em}
}
{
  \end{oldthebibliography}
}
\linespread{0.98}
\parskip 0.1cm
\definecolor{Gray}{gray}{0.6}
\newcommand{\ic}[1]{{\color{red}{#1}}}
\newcommand{\kc}[1]{{\color{blue}{#1}}}




%The next command sets up an environment for the abstract to your paper.

\newenvironment{sciabstract}{%
\begin{quote} \bf}
{\end{quote}}



% Include your paper's title here

\title{Towards a Robotic Ensemble for Ocean Observation} 


% Place the author information here.  Please hand-code the contact
% information and notecalls; do *not* use \footnote commands.  Let the
% author contact information appear immediately below the author names
% as shown.  We would also prefer that you don't change the type-size
% settings shown here.

\author
{John Smith,$^{1\ast}$ Jane Doe,$^{1}$ Joe Scientist$^{2}$\\
\\
\normalsize{$^{1}$Department of Chemistry, University of Wherever,}\\
\normalsize{An Unknown Address, Wherever, ST 00000, USA}\\
\normalsize{$^{2}$Another Unknown Address, Palookaville, ST 99999, USA}\\
\\
\normalsize{$^\ast$To whom correspondence should be addressed; E-mail:  jsmith@wherever.edu.}
}

% Include the date command, but leave its argument blank.

\date{}


\begin{document}

% ***********************************
% Uncomment the following for double space
% \baselineskip24pt
% ***********************************

\maketitle 

% Place your abstract within the special {sciabstract} environment.

\begin{sciabstract}
  The worlds oceans have been changing rapidly for some time; however
  the observational methods and capacity have hewed to the tradition
  of ship based approaches with discrete measurements to discern
  continuous processes in space and time.  The ongoing revolution in
  Robotics, Artificial Intelligence and sensor development are making
  rapid strides in how we are and likely to observe our global
  oceans. We believe it is only a matter of time for technology to
  make a sustained and large scale impact on how we observe the
  complexity of the upper ocean. The trajectory we take will however
  be important; we offer a perspective on an approach we believe will
  help solve the fundamental problem of undersampling with
  multi-domain ensembles of intelligent, networked robots.
  
  % JBS but this is just the beginning
  
  % It would be great to have a few diagrams depicting the structure
  % of the paper % Discussion of technological trends: # of vehicles will
  % increase significantly; new sats such as the ones from spaceX; from
  % vehicle systems to systems of systems; system-like or team-like
  % capabilities (find front; track front); immersive/viz holographic
  % realities; find correlations in data streams; % discussion of the
  % process for the paper: outline; get reactions; re-iterate
\end{sciabstract}

% \setcounter{secnumdepth}{2} 

\section{Introduction}

The world's oceans are a critical component for the habitability of
the 'pale blue dot' imaged by the Voyager mission (in 1990), that we
can call planet earth. Their photic zone with abundant phyto-plankton
produces the oxygen in every other breath we take. The ocean is a
living water mass with dynamic bio-geochemical processes that power
the abundance of plankton at the base of the human food chain.  Since
the industrial revolution, the oceans have absorbed increasing amounts
of anthropogenic CO\textsubscript{2} that humans have been spewing
into the atmosphere. They also provide a lifeline for the worlds
economies, a regulator of the planets weather and climate and also a
distinct source of relaxation and leisure activities. Yet the upper
ocean, arguably a small part of the worlds water-mass and our
interface to the ocean domain, is under-sampled and poorly understood
\cite{munk2002}. The principal problem in ocean science is to
understand the processes which power this part of the ocean and to do so,
we need to understand the ocean not just in space but also in time,
i.e. in 4D. 

Traditional methods pioneered by Charles Darwin with his exploration
of new worlds on the \emph{Beagle}, have continued to drive how we
observe and sample the water-column as a means to understand
bio-geochemical processes typically with water-column measurements in
3D or 2D.  Modern research vessels typically make discrete water
sample measurements instrumented by lowering CTDs rosettes and
plankton nets which require the vessel to stop periodically. These
measurements are pulled together with assumptions about space/time
variability to discern what are continuous processes in space and
time. Underway measurements including those towed behind the research
vessel such as toyo's (\kc{cite}) provide continuous measurements but
are restricted to the vessels movement and have no independant means
to adapt to variability in the water-column. Eulerian and Lagrangian
approaches to observations such as buoys and \texttt{ARGO} floats
\cite{roemmich09} can make accurate assessments with temporal
variability and space and time respectively, but are constrained by
what water mass passes by. Driven by the Oil and Gas industry,
remotely operated vehicles (ROVs) are in a different class in of
themselves. As tethered vehicles, they've been immensely successful in
exploration where \emph{manipulation} rather than measurement is at
the core; this is particularly in the context of benthic exploration
including those involving marine science \cite{yoerger00,robi17} and
archeology \cite{coleman00}. ROVs however are not autonomous and not
the focus of this paper.

More recently, robotic vehicles have been augmenting traditional
ship-based observations, in the aerial, surface and underwater
domains. Simple buoyancy driven gliders, with hydrofoil wings and no
propulsion initiated this transition and have demonstrated sustained
in-water presence \cite{rucool11} as autonomous underwater vehicles
(AUVs). However, their adaptability to measure water-column properties
has been limited to straight line transects. Powered AUVs with
thrusters and larger electrically driven payloads have been making
inroads to this trend \cite{loch89,dorado2004,Bellingham07}. Equally,
with a larger onboard computational capacity, more sophisticated
control systems and algorithms have enabled these vehicles to adapt to
sensory signals from their scientific payload and demonstrate an
information gathering capability which is novel and likely to change
how water-column measurements are made
\cite{bellingham94,aosn93,ryan10,das11b,das15,fossum18,fossum18b}.

Autonomous surface vehicles (ASVs) have till recently directed more
towards security operations \cite{wolf10}. However recent innovations
in energy harvesting have resulted in a range of platforms driven by
waves \cite{waveglider,verfuss19} and wind \cite{gentemann20,ghani14},
have allowed for longer-duration presence on the oceans surface, often
a harsh environment.

While AUVs have become more acceptable in oceanographic observations,
it is interesting to note that both ASVs and unmanned aerial vehicles
(UAVs), have had a harder time gaining acceptance. The latter in
particular have been stymied by battery technology in being
circumscribed to operate near shore or via simple methods in launch
and recovery from a research vessel \cite{Ferreira2018}. Equally
payloads suitable both in mass and energy requirements for carrying on
such ship-launched platforms for over water observations have been
sparse compared to the rapid innovation in sensors for terrestrial
drones. This too however, is changing with low cost DMS sensors (to
measure outgassing of biological productivity) (\kc{cite}) and imagers
including hyper-spectral \cite{sigernes18} and infra-red sensors now
available for integration.

While substantial progress has been made in bringing each class of
these assets to aid and augment ocean observation, some more
effectively than others, less has been done to make effective use of
these vehicles in a sustainable and cohesive manner to actually make a
significant leap in ocean observation. To do so, we believe that in
adopting the following \textbf{four} ways, we can significantly
advance not just the science of ocean observation, but also a
systematically engineered approach to marine robotics.

First, the adoption of \emph{networked} robotics is critical to
increasing the sensing footprint of a research vessel. Making discrete
measurements of macro-phenomenon (e.g. blooms, anoxic zones, plumes of
any kind, fronts) across space and time, requires un-aliased sensing
which requires the presence of multiple sensory elements across a
large (typically on meso-scale $\sim 50$ Km\textsuperscript{2})
spatial extant. While eulerian or lagrangian sensors are possible,
practicality requirements in the open (or even coastal) ocean imply
the use of mobile robotic assets. For their placement and control,
networked methods to communicate and control are critical for
subsequent data assimilation and analysis.

Second, even if multiple robotic platforms (mobile or immobile) are
available and placed in an 'appropriate' location for making the
necessary measurements, if they do not \emph{adapt}, then a large
portion of the signal being tracked is likely to be missed or poorly
resolved, spatio-temporally (\kc{cite}).

Third, to look at phenomenon at scale, both high-resolution spatially
coherent data as also fine-scale temporal resolution is needed
(\kc{cite}) to separate bio-geophysical interactions, for instance in
wind-driven upwelling. \emph{Synoptic} views are therefore important
in resolving the scales of the processes in question, so both the
macro- perspective, provided by remote sensing and the micro view
provided by in-situ assets making continuous measurements are required
to be fused together.

Finally, a mix of predictive and reactive approaches are required to
be able to determine the central problem in oceanographic sampling,
i.e. ``where and when to sample''. And to do understand the
spatio-temporal aspects of this living mass in 4D to to zoom in and
out and map in and out large or smaller volumes and to look at micro
to macro structure. Ocean models driven by remote sensing and highly
non-linear computation can provide hindcasts, nowcasts and
forecasts. These can be valuable in event-response situations such as
oil spills and other forms of coastal marine pollution or natural
forcings that can cause blooms, for instance. However model skill in
making predictions with sufficient levels of confidence continues to
dodge coastal, sub-mesoscale and near-shore waters. Coupling models
with assimilated sensor measurements however, provide a way out to
increasing model skill and making predictions, especially in dynamic
mixed waters in the coastal environment.

To cover these approaches, we need an ensemble of robotic vehicles
with different sensors to be able to characterize different processes
at different scales and varying levels of synopticity. And to have
them coherently looking at one patch of the ocean over space, aerial,
surface and underwater domains and to be able to integrate
these measurements to provide a cogent ``MRI'' scan of the upper
water-column. And to do so, with as much automation in hardware and
platforms, as well as in software data synthesis, analysis and
decision making in an interdisciplinary manner merging ocean science,
robotics and Artificial Intelligence (AI) including Machine Learning
(ML). 


\begin{enumerate} 

%   \item Why do we observe the world's oceans? I.e. importance of the oceans
% themselves

% \item How do we observe?

\item What are its inherent limitations and why scientists do what
  they do; this to address the issues related to variability in the
  water-column, physics (e.g. type 1 vs type 2 regions), seasonal and
  temporal (e.g. tidal) variability. All packed as an oceanography 101
  type of material within a para or two

\item What is the focus of this m/s? Delineate upper water-column from the
meso-pelagic an down to the benthic.

\item A brief (1-2 sentence) history of ocean observation from Darwin to
Challenger and onwards, and the emphasis and need for ships.

\item talk about how ships themselves have evolved in some form
  (e.g. Falkor’s super-computer, and the viability of high-bandwidth
  comms to make remote work feasible (e.g. Bob Ballard’s Ocean Space
  Center and the R/V Nautilus)



\item Pulling apart biological from Phys., Chemical and Geological ocean
observation. 

\item How did Satellite remote sensing change the way ocean science has
been done and what impacts they've had

\end{enumerate}

\section{Robotics and AI in Ocean Observation}

Trace the advent of scientific instrumentation which morphed into
floats, into gliders and powered AUVs.

\begin{enumerate} 

  \item articulate the various 'robotic' vehicles, mobile and
    immobile. Keep this general, so even a buoy is a robotic sensing
    platform

  \item how an ensemble of vehicles can extend the “reach" of the
    human senses onboard the ship and perhaps even from shore with
    high bandwidth comms -- extend the above to not just water-column,
    but benthic work (where I know very little)

  \item Show the figure of which robotic assets are viable for what
    kinds of observation. Overlay bio-physical processes which are
    appropriate and discuss at length why these assets suit those
    specific observations.

  \item Articulate how robots have 'extended the human senses' from
    ship and shore to provide new ways of observing the ocean

  \item In brief -- examples of Machine Learning and other forms of AI
    which can help and how (see below). Machine Learning offline or
    even inline in the perspective of “discovery"

  \item How systematic observation, as against point measurements
    (i.e. dipping a rosette) and using extrapolation, can help. What
    kinds of signals are being missed

  \item Harshness of the environment and operational issues of being
    at sea for sustained presence

\end{enumerate}

\section{Ocean models}

Articulate what models do, how they contribute and what their state of
the art is. Ensure this talks about both the physics and the biology.
\section{On Sampling the water column}

\begin{enumerate} 

\item define sampling in the context of ocean science -- systematic
  measurement of variables over space and time, to be able to
  disambiguate cause-effect relationships

\item the importance of measuring variability in 4D (Space X
  Time). That vertical variability in the water-column is more
  pronounced 
  
\item why is it hard? Static-sensor/static field,
  static-sensor/dynamic field, mobile-sensor platform (AUVs
  etc)/dynamic field. Talk about aliasing of space/time. Refer to
  current methods from the Intro section.

\end{enumerate}

\section{New Horizons -- Future trends in Ocean Science}
\label{sec:future}

The oceans are changing as a consequence of human activity and yet
because our knowledge about this ecosystem is limited, we cannot
accurately model and predict how it will behave in the future.  The
oceans are vast, occupying almost 71\% of our planet’s surface and yet
less than 10\% of it has been studied. The future of maritime
exploration is therefore heavily dependent on automation and robotics.

The vast variability in spatial and temporal scales that need to be
explored has been a consistent challenge to understand the process and
dynamics on which life on earth depends. Consequently, the use of
automation to replace human presence or even its extension, will
require robotic platforms, a diverse range of sensors and control and
analysis algorithms which can stitch a more incisive picture of the
oceanic environment.
% To do so, it becomes critical to view a patch of the
% ocean across spatial and temporal scales to view processes from the
% large to connect with the small (macro to micro). We believe such a
% patch should be at the meso-scale with coordinated observations
% starting from space based, to aerial, surface and underwater to
% discern the processes in the upper water-column so crucial for our
% understanding of what sustains life on earth.
The absence of a reliable, efficient, and rapid monitoring system for
ocean health is greatly impeding our capacity to respond to and
prevent human-induced threats in a timely, context-relevant and
effective way. This is especially important in coastal regions because
these areas mediate most of the interactions between a significant
percentage of the world population and the oceans.  In addition to
other global forcing, urban population growth has exacerbated
pressures on coastal ecosystems resulting in unhealthy alterations
(e.g., toxic algal blooms, oxygen depletion also referred to as
hypoxia) and deleterious effects on fisheries and human health. These
problems are likely to accelerate in the coming decades as extreme
weather events and storm surges will likely enhance agricultural
pollution runoff and coastal erosion, leading to a worsening coastal
water quality.

\pro (Movable ocEan roboTic obsErvatORy) is conceived as a modular
system with bespoke approaches related to water quality in the world’s
coastal zones.  The fully completed observational system will
constitute both a vertical integration of state-of-the-art hardware
including a small satellite (\smle) constellation, as well as in-situ
air, surface and underwater vehicles, with innovative software and
assimilating ocean models to visualize the information gathered and
predict the near term future, as well as horizontal integration across
disciplines of computer science, marine robotics, engineering, risk
quantification, ocean modeling, and oceanography. The frequent revisit
times over a region that only a constellation of \smle s can provide,
coupled with the latest smart and adaptive AI techniques, will enable
robots to deliver systematic and opportune observations of parameters
relevant to coastal water quality and ocean processes in near
real-time.

Fig. \ref{fig:inverse} visually provides such a perspective. Starting
with the \smle, ocean surface observations can be generated with a
\emph{hyper-synoptic} view of the upper water-column at about 10,000
km\textsuperscript{2} in one snap-shot, with a platform moving in the
order of 15,000 knots, with 'higher' resolution data from
\emph{super-synoptic} UAVs making atmospheric and surface measurements
covering $\sim 1000's$ km\textsuperscript{2} flying at 40--60
knots. Closer in and \emph{in-situ}, potentially synoptic measurements
can be made by ASV's on the surface which can make air/sea flux
measurements, covering potentially $\sim 100's$ km\textsuperscript{2} at
potential 2--10 knots speed. Even higher micro-structure measurements
can then be augmented by AUVs with added in-situ imaging and
water-sampling at the $\sim 10's$ km\textsuperscript{2} scale while
moving through the water column between 1--4 knots. Together this
inverse observational pyramid, forms a cohort of platforms with the
sensors they carry, to detect, track and examine features from space,
all the way to the micro-organisms that inhabit those features.

\begin{figure}[!h]
  \centering
  \includegraphics[width=0.9\textwidth]{fig/inverse-pyramid.jpg}
  \caption{Using multi-domain platforms from space, aerial, surface
    and underwater vehicles to observe a patch of the coastal ocean is
    critical to how we can make measurements across the variability in
    spatial and temporal scales. This \textsf{inverse pyramid of
      observational} capability will provide the necessary means to
    discern the bio-geophysical changes in the upper ocean, key to
    human life on Earth.}
  \label{fig:inverse}
\end{figure}

Each platform brings a range of capabilities, from observing at very
large spatial scales via space remote sensing, to making fine scale
measurements in the water column at far slower speeds with AUVs
in-situ. Coordination of these platforms is not necessary about
managing swarming or optimal control behaviors, but more about
ensuring that measurements can be made at approximately the
region/area/volume in as adaptive way as necessary. The expense of
operating such platforms can then work in a dynamic that allows for a
gradated response which use (space or aerial) remote sensing to spot a
feature of interest, which then result in an 'event response'
situation, which can bring to bear more expensive, in terms of
human-aided deployment costs, in-situ assets like ASVs and
AUVs. Aerial platforms can then be used to scour a larger surface area
and direct the deployment of slower moving ASVs and AUVs in targeted
areas. These platforms in turn, can then use onboard adaptive control
to further refine their observations to generate the necessary
fine-scale measurements that science requires at high-resolution. 

\begin{figure}[!h]
  \centering
  \includegraphics[width=0.9\textwidth]{fig/inverse-pyramid-2.jpg}
  \caption{The backend and embedded computation on robotic vehicles in
  the space, aerial, surface and underwater domains are critical for
  the functioning of the ensemble. The cohort of vehicles will need to
  guided by ocean models, Machine Learned systems to direct them to
  areas of maximal uncertainty, methods to assimilate and fuse data
  from multiple sensors and layered visualization which can show a
  range of possible projections and real-time information.}
  \label{fig:inverse-2}
\end{figure}

This hardware-centric view belies the complexity behind such a likely
scenario since it is backend and embedded software that will actually
provide the necessary means to accomplish the tasks to map dynamic
features in the water-column (Fig. \ref{fig:inverse-2}). Central to
the back end will be data-assimilated ocean models which will
assimilated bio-physical measurements from space and aerial remote
sensing, in addition to digesting continuous measurements from
moorings, buoys, drifters, floats and glider lines to generate
uncertainty maps which can target powered robotic vehicles like ASVs
and AUVs to adaptively sample the water column in high resolution
\cite{berget18,fossum18,fossum18b,fossum19b,fossum21}.


% This could be the core of the m/s -- a look ahead to what we think the
% contributions of AI and Robotics can do, leveraging networked vehicle
% technologies, given large spatial extents to be sampled. 

\begin{enumerate} 
\kc{
\item Implications of the use of robotic vehicles -- plusses and
  challenges. The role of vehicles in space, aerial, surface and
  underwater environments

\item how new generations of spacecraft (incl. SmallSats) could alter
  the landscape — e.g. our pitch to Audacious
  
\item How AI/ML can tie the needs of observational requirements and
  alleviate the issue of space/time and understanding spatio-temporal
  cause-effect relationships

\item The use of robots in security and surveillance. Legal implications
  related to use of robotic vehicles in such domains. 
}
\end{enumerate}

  
\section*{Concluding remarks}

% \begin{enumerate} 

% \end{enumerate}




\bibliography{references}

\bibliographystyle{Science}

\section*{Acknowledgments}

\section{Notes}

Notes from July 2nd conversation.

\begin{itemize}[noitemsep,topsep=0pt,parsep=0pt,partopsep=0pt]

\item The purpose of this work is to think of the ocean as a living
  water mass, with dynamic processes in the upper water-column.

\item This part of the water column and the photic zone and the
  productivity of this upper water column has a direct relationship with
  our existence on the planet by generating oxygen. Every other breath
  we take comes from phytoplankton generated oxygen.

\item The principal problem in ocean science is to understand the
  processes which power this part of the ocean. To do so, we need to
  understand the ocean not just in space but also in time, i.e. in
  4D. Typically we have been looking at water-column measurements in 3D
  or 2D.

\item We need to use such 4D visibility to zoom in and out, map in and
  out large or smaller areas.

\item And to augment process studies to enable looking from micro
  structure to the macro to understand the impact of the changing
  oceans.

\item Further, some processes might be visible to one and not the other
  sensor. Consequently, we will need to look at the ocean with multiple
  sensors at multiple resolutions, at varying levels of synopticity.

\item Therefore we need robotic platforms which can cover space and time
  in varying ways, and we need mechanisms to be able to integrate
  measurements across these different sensors and platforms.

\item And to do so, with as much automation as possible, to cover the
  vast stretches of often hostile and harsh oceans.

\item Data fusion and machine ``discovery'' will therefore be an
  important part of this narrative to enable panning thru troves of data
  once sensors and platforms are in place to observe.

\item Discovery will come in part from understanding the nature of
  ``surprise'' in data which is a part of exploration.

\item Understanding of Lagrangian Vs. Eulerian methods of
  observation with constant environmental change. Example, observing a
  changing iceberg from below -- observe the change as the dynamics of
  the moving iceberg and its composition are constantly changing.

\item Using robots is not just about measuring at scale for space X
  time, and also not about measuring specific variables but in
  understanding the specific processes which we need to disambiguate.

\item Calibration of a range of sensors looking at the same patch of the
  ocean is a challenge.

\item Data fusion is another.

\item Need a figure which can be used to layer processes with
  sensors/platforms akin to the one from Scott Doney.

\end{itemize}

\newpage
\large{Be sure to see \url{https://www.sciencemag.org/node/2385628}
  for info on
  Sci. Robotics Perspectives and what they're looking for}\\


Potential folks to recruit, once we have a baseline draft; those in
\ic{red} are likely to contribute text:

\begin{enumerate}[noitemsep,topsep=0pt,parsep=0pt,partopsep=0pt]
\footnotesize{  
\item \ic{Jo Eidsvik} (NO) NTNU, Trondheim -— Statistics and Sampling https://www.ntnu.edu/employees/jo.eidsvik
\item \ic{Rick Stumpf} (US) NOAA — coastal oceanography -- https://www.gulfbase.org/people/dr-richard-p-stumpf
\item Catarina Magalhaes (PO) Univ. of Porto, CIIMAR — biological oceanography -- https://www2.ciimar.up.pt/team.php?id=85
\item Stef Williams (AUS) Univ. of Sydney — Marine Robotics -- https://www.sydney.edu.au/engineering/about/our-people/academic-staff/stefan-williams.html
\item \ic{Ralf Bachmeyer} (DE) Univ. of Bremen, Germany — Marine Robotics -- https://www.marum.de/en/Prof.-Dr.-ralf-bachmayer.html
\item \ic{Yogi Girdhar} (US), WHOI — Marine Robotics and ML -- https://www.whoi.edu/profile/ygirdhar/
\item Paulo Relvas (PO), Univ. of Algarve, Portugal — Phys. Oceanography -- https://www.ccmar.ualg.pt/users/prelvas
\item Marta Chantal Ribeiro (PO) Univ. of Porto, CIIMAR -— Maritime Law and MPA’s -— https://www2.ciimar.up.pt/team.php?id=279
\item Oliver Zelinskly (DE), Oldenberg, Germany -- Sensors and ML -- https://uol.de/en/icbm/marine-sensor-systems
\item Nadia Pinardi (IT) Bologna, Italy -- Phys. Oceanography -- https://www.unibo.it/sitoweb/nadia.pinardi/en
\item \ic{Oscar Scholfield} (US) Rutgers -- biological oceanographer -- https://marine.rutgers.edu/team/oscar-schofield/  
\item Mark Moline (US) Univ. of Delaware -- biological oceanographer -- https://www.udel.edu/academics/colleges/ceoe/departments/smsp/faculty/mark-moline/
\item Pere Ridao (ES) Univ. of Girona -- Marine Robotics -- http://eia.udg.es/~pere/Pere\_Ridao\_Home\_Page/Short\_CV.html
\item Mandar Chitre (SG) National Univ. Singapore -- Marine Robotics
  and acoustics -- http://www.chitre.net/

}
\end{enumerate}


\end{document} 
