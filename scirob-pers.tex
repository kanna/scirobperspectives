\documentclass[12pt]{article}

% The preamble here sets up a lot of new/revised commands and
% environments.  It's annoying, but please do *not* try to strip these
% out into a separate .sty file (which could lead to the loss of some
% information when we convert the file to other formats).  Instead, keep
% them in the preamble of your main LaTeX source file.
\usepackage[font=footnotesize]{caption}
\usepackage{float}
\usepackage{epsf}
\usepackage{epsfig}
\usepackage{subfigure}
% \usepackage{subfig}
\usepackage{latexsym}
%\usepackage{algorithm}
%\usepackage[noend]{algorithmic}
% \usepackage{color}
\usepackage{color, colortbl}
\usepackage{wrapfig}
\usepackage{topcapt}
\usepackage{multirow}
\usepackage{tabularx}
\usepackage{hyperref}
\usepackage{xcolor}
\usepackage{mdwlist}


%\usepackage{color}
\newcommand{\lc}[1]{\textcolor{blue}{#1}}


% \usepackage{amsmath, amsfonts, amssymb}
% \usepackage{algorithmic}
\usepackage[hmargin=1in,vmargin=1.2in]{geometry}
\usepackage{url}
\usepackage{multirow}
\usepackage[ruled,noline,linesnumbered]{algorithm2e}
\usepackage[bottom]{footmisc}
\usepackage{afterpage}
%\usepackage[caption=false]{caption}
\usepackage{eurosym}
\usepackage{enumitem}
% \usepackage{soul}
\usepackage{fancyhdr}
\usepackage{dashrule}
% \usepackage[stable]{footmisc}
% \usepackage{placeins}
% \setitemize{noitemsep,topsep=0pt,parsep=0pt,partopsep=0pt}
% \usepackage[utf8]{inputenc}
\usepackage{gensymb}
\usepackage{rotating}
\usepackage{tikz}


\widowpenalty=1000
\clubpenalty=1000

\def\rx{{\texttt{T-REX\ }}}
\def\rxe{{\texttt{T-REX}}}
\def\ls{{\texttt{LSTS\ }}}
\def\lse{{\texttt{LSTS}}}
\def\eut{{\texttt{EUROPA}$_2$\ }}
\def\eu{{\texttt{EUROPA}\ }}
\def\eue{{\texttt{EUROPA}}}
\def\eus{{\texttt{EUROPA}'s\ }}
\def\nd{{\texttt{NDDL\ }}}
\def\nde{{\texttt{NDDL}}}
\def\pd{{\texttt{PDDL\ }}}
\def\pde{{\texttt{PDDL}}}
\def\eur{{\texttt{EUROPtus\ }}}
\def\eure{{\texttt{EUROPtus}}}
\def\nas{{\texttt{NASA\ }}}
\def\nase{{\texttt{NASA}}}
\def\sml{{\texttt{SmallSat\ }}}
\def\smle{{\texttt{SmallSat}}}
\def\univ{{\texttt{UPorto\ }}}
\def\unive{{\texttt{UPorto}}}
\def\aut{{\texttt{Autonaut\ }}}
\def\aute{{\texttt{Autonaut}}}
\def\proj{{\textbf{PROTEAN\ }}}
\def\proje{{\textbf{PROTEAN}}}
\def\col{{\texttt{COLREGS\ }}}
\def\cole{{\texttt{COLREGS}}}

\def\etal{{et al.\/}}
\def\eg{e.g., }
\def\ie{{i.e.,\ }}
\def\etc{{etc.\ }}
\def\situ{{in situ \/}}
\def\PN{{\emph{PN} }}


\input{epsf}

%\usepackage{mathptmx}
%\usepackage{multirow}

\newcommand{\rtime}[1]{\par\noindent\rlap{#1} \hspace*{2.15cm}}
\newcommand{\iblank}{\par \noindent \hspace*{2.4cm} \hangindent 2.6cm}
\newcommand{\m}[1]{\ensuremath{\mathbf{#1}}}
\newcommand{\mc}[1]{\ensuremath{\mathcal{#1}}}
% \newcommand{\mb}[1]{\mbox{\boldmath$#1$\unboldmath}}
%\newcommand{\norm}[1]{\left| \left| #1 \right| \right| ^2}
\newcommand{\snr}{\hbox{SNR}}
\newcommand{\mse}{\hbox{MSE}}
\newcommand{\E}{{\mathbb E}}
\newcommand{\cn}{{\mathcal{CN}}}
\newcommand{\ba}{\begin{align*}}
\newcommand{\ea}{\end{align*}}

\newcommand{\real}{{\mathbb{R}}}
\newcommand{\integer}{{\mathbb{Z}}}
\renewcommand{\natural}{{\mathbb{N}}}
\newcommand{\argmin}{\operatorname{argmin}\displaylimits}
\newcommand{\argmax}{\operatorname{argmax}\displaylimits}

\newcommand{\relthresh}{{T_{\text{rel}}}}
\newcommand{\absthresh}{{T_{\text{abs}}}}

\newcommand{\nprof}{{N_{\text{prof}}}}
\newcommand{\DM}{{DM}}
\newcommand{\UM}{{UM}}
\newcommand{\deltaMax}{{\partial_{\max}}}
\newcommand{\IFD}{{IFD}}
\newcommand{\IFU}{{IFU}}

\newcommand{\mvdiff}{\mathbf{mvd}}
\newcommand{\mvest}{\widehat{\mvdiff}}
\newcommand{\prof}{p}

\newtheorem{Prop}{Proposition}
\newtheorem{Theorem}{Theorem}
\newtheorem{Lemma}{Lemma}
\newtheorem{Corrolary}{Corollary}

\def\be{\begin{equation}}
\def\ee{\end{equation}}

\newlength{\doublespacelength}
\setlength{\doublespacelength}{\baselineskip}
\addtolength{\doublespacelength}{0.5\baselineskip}
\newcommand{\doublespace}{\setlength{\baselineskip}{\doublespacelength}}

\newlength{\singlespacelength}
\setlength{\singlespacelength}{\baselineskip}
\newcommand{\singlespace}{\setlength{\baselineskip}{\singlespacelength}}


\newlength{\savedspacing}
\newcommand{\savespacing}{\setlength{\savedspacing}{\baselineskip}}
\newcommand{\restorespacing}{\setlength{\baselineskip}{\savedspacing}}

\setlength{\parskip}{0pt}
\setlength{\parsep}{0pt}
\setlength{\headsep}{0pt}
\setlength{\topskip}{0pt}
\setlength{\topmargin}{0pt}
\setlength{\topsep}{0pt}
\setlength{\partopsep}{0pt}
% \setlength{\parindent}{0pt}

\newcommand{\icomnt}[1]{{\color{red}{#1}}}
\newcommand{\kcomnt}[1]{{\color{blue}{#1}}}

\newcommand{\unit}[1]{\ensuremath{\mathrm{#1}}}                  %%%% to units and other roman math stuff
% \linespread{0.98}
% % \linespread{2.00}

\newcounter{quotenumber}

\newenvironment{numquote}{%
    \begin{enumerate}%
     \setcounter{enumi}{\value{quotenumber}}%
     \color{darkgray}
    \item \begin{quote}%
}{%
    \end{quote}%
    \setcounter{quotenumber}{\value{enumi}}
    \end{enumerate}%
}%

\makeatletter
\def\myitem{%
   \@ifnextchar[ \@myitem{\@noitemargtrue\@myitem[\@itemlabel]}}
\def\@myitem[#1]{\item[#1]\mbox{}}
\makeatother



\newcommand\blankpage{%
    \null
    \thispagestyle{empty}%
    \addtocounter{page}{-1}%
    \newpage}

\setcounter{secnumdepth}{0} 

\let\oldthebibliography\thebibliography
\let\endoldthebibliography\endthebibliography
\renewenvironment{thebibliography}[1]{
  \begin{oldthebibliography}{#1}
    \setlength{\itemsep}{0em}
    \setlength{\parskip}{0em}
}
{
  \end{oldthebibliography}
}
\linespread{0.98}
\parskip 0.1cm
\definecolor{Gray}{gray}{0.6}



%The next command sets up an environment for the abstract to your paper.

\newenvironment{sciabstract}{%
\begin{quote} \bf}
{\end{quote}}



% Include your paper's title here

\title{Towards a Robotic Ensemble for Ocean Observation} 


% Place the author information here.  Please hand-code the contact
% information and notecalls; do *not* use \footnote commands.  Let the
% author contact information appear immediately below the author names
% as shown.  We would also prefer that you don't change the type-size
% settings shown here.

\author
{John Smith,$^{1\ast}$ Jane Doe,$^{1}$ Joe Scientist$^{2}$\\
\\
\normalsize{$^{1}$Department of Chemistry, University of Wherever,}\\
\normalsize{An Unknown Address, Wherever, ST 00000, USA}\\
\normalsize{$^{2}$Another Unknown Address, Palookaville, ST 99999, USA}\\
\\
\normalsize{$^\ast$To whom correspondence should be addressed; E-mail:  jsmith@wherever.edu.}
}

% Include the date command, but leave its argument blank.

\date{}


\begin{document}

\baselineskip24pt

% Make the title.

\maketitle 

% Place your abstract within the special {sciabstract} environment.

\begin{sciabstract}
  The worlds oceans have been changing rapidly for some time; however
  the observational methods and capacity have hewed to the tradition
  of ship based approaches with discrete measurements to discern
  continuous processes in space and time.  The ongoing revolution in
  Robotics, Artificial Intelligence and sensor development are making
  rapid strides in how we are and likely to observe our global
  oceans. 
\end{sciabstract}

% \setcounter{secnumdepth}{2} 


\section{Introduction}


\begin{enumerate} 

  \item Why do we observe the world's oceans? I.e. importance of the oceans
themselves

\item How do we observe?

\item What are its inherent limitations and why scientists do what
  they do; this to address the issues related to variability in the
  water-column, physics (e.g. type 1 vs type 2 regions), seasonal and
  temporal (e.g. tidal) variability. All packed as an oceanography 101
  type of material within a para or two

\item What is the focus of this m/s? Delineate upper water-column from the
meso-pelagic an down to the benthic.

\item A brief (1-2 sentence) history of ocean observation from Darwin to
Challenger and onwards, and the emphasis and need for ships.

\item talk about how ships themselves have evolved in some form
  (e.g. Falkor’s super-computer, and the viability of high-bandwidth
  comms to make remote work feasible (e.g. Bob Ballard’s Ocean Space
  Center and the R/V Nautilus)



\item Pulling apart biological from Phys., Chemical and Geological ocean
observation. 

\item How did Satellite remote sensing change the way ocean science has
been done and what impacts they've had

\end{enumerate}

\section{Robotics and AI in Ocean Observation}

Trace the advent of scientific instrumentation which morphed into
floats, into gliders and powered AUVs.

\begin{enumerate} 

  \item articulate the various 'robotic' vehicles, mobile and
    immobile. Keep this general, so even a buoy is a robotic sensing
    platform

  \item how an ensemble of vehicles can extend the “reach" of the
    human senses onboard the ship and perhaps even from shore with
    high bandwidth comms -- extend the above to not just water-column,
    but benthic work (where I know very little)

  \item Show the figure of which robotic assets are viable for what
    kinds of observation. Overlay bio-physical processes which are
    appropriate and discuss at length why these assets suit those
    specific observations.

  \item Articulate how robots have 'extended the human senses' from
    ship and shore to provide new ways of observing the ocean

  \item In brief -- examples of Machine Learning and other forms of AI
    which can help and how (see below). Machine Learning offline or
    even inline in the perspective of “discovery"

  \item How systematic observation, as against point measurements
    (i.e. dipping a rosette) and using extrapolation, can help. What
    kinds of signals are being missed

  \item Harshness of the environment and operational issues of being
    at sea for sustained presence

\end{enumerate}

  
\section{On Sampling the water column}

\begin{enumerate} 

\item define sampling in the context of ocean science -- systematic
  measurement of variables over space and time, to be able to
  disambiguate cause-effect relationships

\item the importance of measuring variability in 4D (Space X
  Time). That vertical variability in the water-column is more
  pronounced 
  
\item why is it hard? Static-sensor/static field,
  static-sensor/dynamic field, mobile-sensor platform (AUVs
  etc)/dynamic field. Talk about aliasing of space/time. Refer to
  current methods from the Intro section.

\end{enumerate}

\section{New Horizons -- Bringing the two together -- Future trends in Ocean Science}

This could be the core of the m/s -- a look ahead to what we think the
contributions of AI and Robotics can do, leveraging networked vehicle
technologies, given large spatial extents to be sampled. 

\begin{enumerate} 

\item Implications of the use of robotic vehicles -- plusses and
  challenges. The role of vehicles in space, aerial, surface and
  underwater environments

\item how new generations of spacecraft (incl. SmallSats) could alter
  the landscape — e.g. our pitch to Audacious
  
\item How AI/ML can tie the needs of observational requirements and
  alleviate the issue of space/time and understanding spatio-temporal
  cause-effect relationships

\item The use of robots in security and surveillance. Legal implications
  related to use of robotic vehicles in such domains. 

\end{enumerate}


\section*{Concluding remarks}

% \begin{enumerate} 

% \end{enumerate}


\bibliography{references}

\bibliographystyle{Science}

\section*{Acknowledgments}

\section{Notes}

Notes from July 2nd conversation.

\begin{itemize}[noitemsep,topsep=0pt,parsep=0pt,partopsep=0pt]

\item The purpose of this work is to think of the ocean as a living
  water mass, with dynamic processes in the upper water-column.

\item This part of the water column and the photic zone and the
  productivity of this upper water column has a direct relationship with
  our existence on the planet by generating oxygen. Every other breath
  we take comes from phytoplankton generated oxygen.

\item The principal problem in ocean science is to understand the
  processes which power this part of the ocean. To do so, we need to
  understand the ocean not just in space but also in time, i.e. in
  4D. Typically we have been looking at water-column measurements in 3D
  or 2D.

\item We need to use such 4D visibility to zoom in and out, map in and
  out large or smaller areas.

\item And to augment process studies to enable looking from micro
  structure to the macro to understand the impact of the changing
  oceans.

\item Further, some processes might be visible to one and not the other
  sensor. Consequently, we will need to look at the ocean with multiple
  sensors at multiple resolutions, at varying levels of synopticity.

\item Therefore we need robotic platforms which can cover space and time
  in varying ways, and we need mechanisms to be able to integrate
  measurements across these different sensors and platforms.

\item And to do so, with as much automation as possible, to cover the
  vast stretches of often hostile and harsh oceans.

\item Data fusion and machine ``discovery'' will therefore be an
  important part of this narrative to enable panning thru troves of data
  once sensors and platforms are in place to observe.

\item Discovery will come in part from understanding the nature of
  ``surprise'' in data which is a part of exploration.

\item Understanding of Lagrangian Vs. Eularian methods of
  observation with constant environmental change. Example, observing a
  changing iceberg from below -- observe the change as the dynamics of
  the moving iceberg and its composition are constantly changing.

\item Using robots is not just about measuring at scale for space X
  time, and also not about measuring specific variables but in
  understanding the specific processes which we need to disambiguate.

\item Calibration of a range of sensors looking at the same patch of the
  ocean is a challenge.

\item Data fusion is another.

\item Need a figure which can be used to layer processes with
  sensors/platforms akin to the one from Scott Doney.

\end{itemize}

\newpage
\large{Be sure to see \url{https://www.sciencemag.org/node/2385628}
  for info on
  Sci. Robotics Perspectives and what they're looking for}\\


Potential folks to recruit, once we have a baseline draft:

\begin{enumerate}[noitemsep,topsep=0pt,parsep=0pt,partopsep=0pt]
\footnotesize{  
\item Jo Eidsvik (NO) NTNU, Trondheim -— Statistics and Sampling https://www.ntnu.edu/employees/jo.eidsvik
\item Rick Stumpf (US) NOAA — coastal oceanography -- https://www.gulfbase.org/people/dr-richard-p-stumpf
\item Catarina Magalhaes (PO) Univ. of Porto, CIIMAR — biological oceanography -- https://www2.ciimar.up.pt/team.php?id=85
\item Stef Williams (AUS) Univ. of Sydney — Marine Robotics -- https://www.sydney.edu.au/engineering/about/our-people/academic-staff/stefan-williams.html
\item Ralf Bachmeyer (DE) Univ. of Bremen, Germany — Marine Robotics -- https://www.marum.de/en/Prof.-Dr.-ralf-bachmayer.html
\item Yogi Girdhar (US), WHOI — Marine Robotics and ML -- https://www.whoi.edu/profile/ygirdhar/
\item Paulo Relvas (PO), Univ. of Algarve, Portugal — Phys. Oceanography -- https://www.ccmar.ualg.pt/users/prelvas
\item Marta Chantal Ribeiro (PO) Univ. of Porto, CIIMAR -— Maritime Law and MPA’s -— https://www2.ciimar.up.pt/team.php?id=279
\item Oliver Zelinskly (DE), Oldenberg, Germany -- Sensors and ML -- https://uol.de/en/icbm/marine-sensor-systems
\item Nadia Pinardo (IT) Bologna, Italy -- Phys. Oceanography -- https://www.unibo.it/sitoweb/nadia.pinardi/en
\item Oscar Scholfield (US) Rutgers -- biological oceanographer -- https://marine.rutgers.edu/team/oscar-schofield/  
\item Mark Moline (US) Univ. of Delaware -- biological oceanographer -- https://www.udel.edu/academics/colleges/ceoe/departments/smsp/faculty/mark-moline/
\item Pere Ridao (ES) Univ. of Girona -- Marine Robotics -- http://eia.udg.es/~pere/Pere\_Ridao\_Home\_Page/Short\_CV.html
\item Mandar Chitre (SG) National Univ. Singapore -- Marine Robotics
  and acoustics -- http://www.chitre.net/

}
\end{enumerate}


\end{document} 
