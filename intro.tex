\section{Introduction}

The world's oceans are a critical component for the habitability of
the 'pale blue dot' imaged by the Voyager mission (in 1990), that we
can call planet earth. Their photic zone with abundant phyto-plankton
produces the oxygen in every other breath we take. The ocean is a
living water mass with dynamic bio-geochemical processes that power
the abundance of plankton at the base of the human food chain.  Since
the industrial revolution, the oceans have absorbed increasing amounts
of anthropogenic CO\textsubscript{2} that humans have been spewing
into the atmosphere. They also provide a lifeline for the worlds
economies, a regulator of the planets weather and climate and also a
distinct source of relaxation and leisure activities. Yet the upper
ocean, arguably a small part of the worlds water-mass and our
interface to the ocean domain, is under-sampled and poorly understood
\cite{munk2002}. The principal problem in ocean science is to
understand the processes which power this part of the ocean and to do so,
we need to understand the ocean not just in space but also in time,
i.e. in 4D. 

Traditional methods pioneered by Charles Darwin with his exploration
of new worlds on the \emph{Beagle}, have continued to drive how we
observe and sample the water-column as a means to understand
bio-geochemical processes typically with water-column measurements in
3D or 2D.  Modern research vessels typically make discrete water
sample measurements instrumented by lowering CTDs rosettes and
plankton nets which require the vessel to stop periodically. These
measurements are pulled together with assumptions about space/time
variability to discern what are continuous processes in space and
time. Underway measurements including those towed behind the research
vessel such as toyo's (\kc{cite}) provide continuous measurements but
are restricted to the vessels movement and have no independant means
to adapt to variability in the water-column. Eulerian and Lagrangian
approaches to observations such as buoys and \texttt{ARGO} floats
\cite{roemmich09} can make accurate assessments with temporal
variability and space and time respectively, but are constrained by
what water mass passes by. Driven by the Oil and Gas industry,
remotely operated vehicles (ROVs) are in a different class in of
themselves. As tethered vehicles, they've been immensely successful in
exploration where \emph{manipulation} rather than measurement is at
the core; this is particularly in the context of benthic exploration
including those involving marine science \cite{yoerger00,robi17} and
archeology \cite{coleman00}. ROVs however are not autonomous and not
the focus of this paper.

More recently, robotic vehicles have been augmenting traditional
ship-based observations, in the aerial, surface and underwater
domains. Simple buoyancy driven gliders, with hydrofoil wings and no
propulsion initiated this transition and have demonstrated sustained
in-water presence \cite{rucool11} as autonomous underwater vehicles
(AUVs). However, their adaptability to measure water-column properties
has been limited to straight line transects. Powered AUVs with
thrusters and larger electrically driven payloads have been making
inroads to this trend \cite{loch89,dorado2004,Bellingham07}. Equally,
with a larger onboard computational capacity, more sophisticated
control systems and algorithms have enabled these vehicles to adapt to
sensory signals from their scientific payload and demonstrate an
information gathering capability which is novel and likely to change
how water-column measurements are made
\cite{bellingham94,aosn93,ryan10,das11b,das15,fossum18,fossum18b}.

Autonomous surface vehicles (ASVs) have till recently directed more
towards security operations \cite{wolf10}. However recent innovations
in energy harvesting have resulted in a range of platforms driven by
waves \cite{waveglider,verfuss19} and wind \cite{gentemann20,ghani14},
have allowed for longer-duration presence on the oceans surface, often
a harsh environment.

While AUVs have become more acceptable in oceanographic observations,
it is interesting to note that both ASVs and unmanned aerial vehicles
(UAVs), have had a harder time gaining acceptance. The latter in
particular have been stymied by battery technology in being
circumscribed to operate near shore or via simple methods in launch
and recovery from a research vessel \cite{Ferreira2018}. Equally
payloads suitable both in mass and energy requirements for carrying on
such ship-launched platforms for over water observations have been
sparse compared to the rapid innovation in sensors for terrestrial
drones. This too however, is changing with low cost DMS sensors (to
measure outgassing of biological productivity) (\kc{cite}) and imagers
including hyper-spectral \cite{sigernes18} and infra-red sensors now
available for integration.

While substantial progress has been made in bringing each class of
these assets to aid and augment ocean observation, some more
effectively than others, less has been done to make effective use of
these vehicles in a sustainable and cohesive manner to actually make a
significant leap in ocean observation. To do so, we believe that in
adopting the following \textbf{four} ways, we can significantly
advance not just the science of ocean observation, but also a
systematically engineered approach to marine robotics.

First, the adoption of \emph{networked} robotics is critical to
increasing the sensing footprint of a research vessel. Making discrete
measurements of macro-phenomenon (e.g. blooms, anoxic zones, plumes of
any kind, fronts) across space and time, requires un-aliased sensing
which requires the presence of multiple sensory elements across a
large (typically on meso-scale $\sim 50$ Km\textsuperscript{2})
spatial extant. While eulerian or lagrangian sensors are possible,
practicality requirements in the open (or even coastal) ocean imply
the use of mobile robotic assets. For their placement and control,
networked methods to communicate and control are critical for
subsequent data assimilation and analysis.

Second, even if multiple robotic platforms (mobile or immobile) are
available and placed in an 'appropriate' location for making the
necessary measurements, if they do not \emph{adapt}, then a large
portion of the signal being tracked is likely to be missed or poorly
resolved, spatio-temporally (\kc{cite}).

Third, to look at phenomenon at scale, both high-resolution spatially
coherent data as also fine-scale temporal resolution is needed
(\kc{cite}) to separate bio-geophysical interactions, for instance in
wind-driven upwelling. \emph{Synoptic} views are therefore important
in resolving the scales of the processes in question, so both the
macro- perspective, provided by remote sensing and the micro view
provided by in-situ assets making continuous measurements are required
to be fused together.

Finally, a mix of predictive and reactive approaches are required to
be able to determine the central problem in oceanographic sampling,
i.e. ``where and when to sample''. And to do understand the
spatio-temporal aspects of this living mass in 4D to to zoom in and
out and map in and out large or smaller volumes and to look at micro
to macro structure. Ocean models driven by remote sensing and highly
non-linear computation can provide hindcasts, nowcasts and
forecasts. These can be valuable in event-response situations such as
oil spills and other forms of coastal marine pollution or natural
forcings that can cause blooms, for instance. However model skill in
making predictions with sufficient levels of confidence continues to
dodge coastal, sub-mesoscale and near-shore waters. Coupling models
with assimilated sensor measurements however, provide a way out to
increasing model skill and making predictions, especially in dynamic
mixed waters in the coastal environment.

To cover these approaches, we need an ensemble of robotic vehicles
with different sensors to be able to characterize different processes
at different scales and varying levels of synopticity. And to have
them coherently looking at one patch of the ocean over space, aerial,
surface and underwater domains and to be able to integrate
these measurements to provide a cogent ``MRI'' scan of the upper
water-column. And to do so, with as much automation in hardware and
platforms, as well as in software data synthesis, analysis and
decision making in an interdisciplinary manner merging ocean science,
robotics and Artificial Intelligence (AI) including Machine Learning
(ML). 


\begin{enumerate} 

%   \item Why do we observe the world's oceans? I.e. importance of the oceans
% themselves

% \item How do we observe?

\item What are its inherent limitations and why scientists do what
  they do; this to address the issues related to variability in the
  water-column, physics (e.g. type 1 vs type 2 regions), seasonal and
  temporal (e.g. tidal) variability. All packed as an oceanography 101
  type of material within a para or two

\item What is the focus of this m/s? Delineate upper water-column from the
meso-pelagic an down to the benthic.

\item A brief (1-2 sentence) history of ocean observation from Darwin to
Challenger and onwards, and the emphasis and need for ships.

\item talk about how ships themselves have evolved in some form
  (e.g. Falkor’s super-computer, and the viability of high-bandwidth
  comms to make remote work feasible (e.g. Bob Ballard’s Ocean Space
  Center and the R/V Nautilus)



\item Pulling apart biological from Phys., Chemical and Geological ocean
observation. 

\item How did Satellite remote sensing change the way ocean science has
been done and what impacts they've had

\end{enumerate}
