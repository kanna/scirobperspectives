Notes from July 2nd conversation.

\begin{itemize}[noitemsep,topsep=0pt,parsep=0pt,partopsep=0pt]

\item The purpose of this work is to think of the ocean as a living
  water mass, with dynamic processes in the upper water-column.

\item This part of the water column and the photic zone and the
  productivity of this upper water column has a direct relationship with
  our existence on the planet by generating oxygen. Every other breath
  we take comes from phytoplankton generated oxygen.

\item The principal problem in ocean science is to understand the
  processes which power this part of the ocean. To do so, we need to
  understand the ocean not just in space but also in time, i.e. in
  4D. Typically we have been looking at water-column measurements in 3D
  or 2D.

\item We need to use such 4D visibility to zoom in and out, map in and
  out large or smaller areas.

\item And to augment process studies to enable looking from micro
  structure to the macro to understand the impact of the changing
  oceans.

\item Further, some processes might be visible to one and not the other
  sensor. Consequently, we will need to look at the ocean with multiple
  sensors at multiple resolutions, at varying levels of synopticity.

\item Therefore we need robotic platforms which can cover space and time
  in varying ways, and we need mechanisms to be able to integrate
  measurements across these different sensors and platforms.

\item And to do so, with as much automation as possible, to cover the
  vast stretches of often hostile and harsh oceans.

\item Data fusion and machine ``discovery'' will therefore be an
  important part of this narrative to enable panning thru troves of data
  once sensors and platforms are in place to observe.

\item Discovery will come in part from understanding the nature of
  ``surprise'' in data which is a part of exploration.

\item Understanding of Lagrangian Vs. Eulerian methods of
  observation with constant environmental change. Example, observing a
  changing iceberg from below -- observe the change as the dynamics of
  the moving iceberg and its composition are constantly changing.

\item Using robots is not just about measuring at scale for space X
  time, and also not about measuring specific variables but in
  understanding the specific processes which we need to disambiguate.

\item Calibration of a range of sensors looking at the same patch of the
  ocean is a challenge.

\item Data fusion is another.

\item Need a figure which can be used to layer processes with
  sensors/platforms akin to the one from Scott Doney.

\end{itemize}